\usepackage{booktabs} % format tables
\usepackage{times} % times font can reduce page length
\usepackage{color} % color figures
\usepackage{balance} %balance the end of the draft
\usepackage{url} 
%\usepackage[all]{nowidow}
\usepackage{tabularx}
\usepackage{siunitx} %unit package for special unit symbols
\usepackage{epsfig,tabularx,subfigure,multirow, graphicx}
\usepackage{enumitem}

%\usepackage[sorting=none]{biblatex}

\usepackage[bookmarks=false]{hyperref} % add hyper reference to citations and references

\let\proof\relax
\let\endproof\relax
\usepackage{amsthm,amssymb,amsmath}  % support for mathematic symbols

\newcolumntype{L}[1]{>{\raggedright\arraybackslash}p{#1}}
\newcolumntype{C}[1]{>{\centering\arraybackslash}p{#1}}
\newcolumntype{R}[1]{>{\raggedleft\arraybackslash}p{#1}}

\usepackage[linesnumbered,ruled,vlined]{algorithm2e}
\SetKwRepeat{Do}{do}{while}% add do-while sentences
\newcommand\mycommfont[1]{\scriptsize\ttfamily{#1}} %adjust the font size of comments
\SetCommentSty{mycommfont}


\newcommand{\stitle}[1]{\vspace{1ex} \noindent{\bf #1}}
\long\def\comment#1{}

\setlength{\algomargin}{1em} %adjust the side margin of algorithm blocks
%\setlength{\textfloatsep}{1ex} %adjust the space above and below the float block, including algorithms, figures, tables

\DeclareMathOperator*{\argmin}{arg\,min}

\newcommand{\nop}[1]{}

\newcommand{\figureTopMargin}{\vspace{-3ex}}
\newcommand{\figureCaptionMargin}{\vspace{-2ex}}
\newcommand{\figureBelowMargin}{\vspace{-2ex}}
\newcommand{\tableTopMargin}{\vspace{-4ex}}
\newcommand{\tableBelowMargin}{\vspace{-3ex}}
\newcommand{\algoTopMargin}{\vspace{-3ex}}
\newcommand{\algoBelowMargin}{\vspace{-3ex}}
\newcommand{\itemMargin}{\vspace{-1ex}}

\newcommand\abs[1]{\left\lvert #1 \right\rvert}


%%%%%%%uncomment below commands to enable related blocks
\newtheorem{theorem}{\bf Theorem}[section]
\newtheorem{lemma}{\bf Lemma}[section]
\newtheorem{corollary}{\bf Corollary}[section]
\newtheorem{example}{\bf Example}

\theoremstyle{remark}
\newtheorem{remark}{\bf Remark}[section]

\theoremstyle{definition}
\newtheorem{definition}{\bf Definition}

