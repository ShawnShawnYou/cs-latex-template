\section{Solution1}
\label{sec:solution1}

Paragraph 1: what is the general idea of the solution1? Greedy based? DP? Generally introduce solution1.

\subsection{some properties}
Do you have any properties about the problem to use to develop the algorithm? Show them in this subsection with lemmas. 

Or you need define some special values to ease your algorithm description, do it here!


You may need to write some equations. Here I show some example equations styles.

1. Multiple Equations with numbering:

\begin{eqnarray}
A &=& B \label{eq:equation1}\\
B &=& C.\label{eq:equation2}
\end{eqnarray}

2. Single equation with numbering:

\begin{equation}
A = B \label{eq:equation3}
\end{equation}

3. Simple equation without numbering:
$$A = B$$

4. Align equations:
\begin{eqnarray}
&& function(A) \notag\\
&=& A^3 +  A^2 + A^1\label{eq:equation4}
\end{eqnarray}

5. Equation with more than one conditions:

\begin{equation}
A=\left\{
\begin{array}{ll}
A + 1, & A \neq 0 \\
A, & A = 0
\end{array}
\right. \label{eq:equation5}
\end{equation}

6. Linear Programming:

\begin{alignat}{2}
& \text{max}&  & \sum\lambda_{ik}\cdot x_{ik}\\
& \text{s.t.}&    \quad & 
\begin{aligned}[t]
&d(u_i, v)\cdot x_{ik} \leq r,    &&i =1, \dots, m; k = 1, \dots, q,\\
&\sum_{k=1}^{q} x_{ik} \leq 1,					 &&i =1, \dots, m,
\end{aligned}\notag
\end{alignat}

\subsection{proposed algorithm}

Your algorithm needs a interesting name! Stop simply calling it "the Greedy algorithm" or "the Dynamic Programming Based Algorithm"! Do not use these boring names, please!

Introduce the details of your algorithms with natural language. Below is an example of  pseudocode.


\begin{algorithm}[t]
	\DontPrintSemicolon
	\KwIn{A set $C$ of $n$ workers, and a set $R$ of $m$ riders}
	\KwOut{A set of updated scheduling sequences $S$}
	\ForEach{$r_i \in R$}{
		retrieve a list $C_i$ of workers that are valid to $r_i$\;
	}
	
	
	\While{$C_i \neq \emptyset$}{
		
		\uIf{rider $r_i$ can be arranged in $c_j$}{
			do if.\;
			\textbf{break}\;
		}
		\ElseIf{$r_i$ can replace rider $r'_i$ of $c_j$}{
			do else.\;
			\textbf{break}\;
		}
	}
	
		
	

	\Do{condition}{Eample of do-while-loop}
	
	\Return{$\mathbb{S}$}\;
	\caption{ExampleAlgorithm}
	\label{algo:example_algorithm}
\end{algorithm}

\subsection{theory analyses}

Analyze the approximation ratios, competitive ratios, time complexities here.

